\documentclass[a4paper,12pt]{article}
\usepackage[ngerman]{babel}
\usepackage[utf8]{inputenc}
\usepackage{pst-barcode}
\usepackage{booktabs}
\usepackage{tikz}
\usepackage{marginnote}
\usepackage{lipsum}
\usepackage{graphicx}
\usepackage{epstopdf}
\usepackage{amsmath, tabu}
\usepackage{amsfonts}
\usepackage{geometry}
\geometry{
	headsep=0.5cm,
	headheight=2.5cm,
	marginparwidth=2cm,
	textheight=22cm,
}
\setlength{\headheight}{120pt}
\usepackage{fancyhdr}
\setlength\parindent{0pt}
\renewcommand{\familydefault}{\sfdefault}
\renewcommand{\arraystretch}{1.2}
\allowdisplaybreaks
\graphicspath{{../figures/}}


\pagestyle{fancy}
\fancyhf{}

\lhead{
	\textit{Humboldt-Universität zu Berlin, Institut der Informatik}\\
	\bigskip
	\textbf{\Large{Grundlagen der Signalverarbeitung}}\\
	\bigskip
	Abgabe: 19.12.2018\\
	Blatt 09\\[-1em]
}
\rhead{
	Karadeniz, Serra 572423\\
	Kewe, Nina       558403\\
	Pohl, Oliver 	 577878\\
	Zierle, Marc 	 581027\\[-1em]
	\marginnote{\psbarcode[rotate=90,transx=1.75]{572423}{includetext height=0.5}{code39ext}}[1cm]
	\marginnote{\psbarcode[rotate=90,transx=1.75]{558403}{includetext height=0.5}{code39ext}}[6.5cm]
	\marginnote{\psbarcode[rotate=90,transx=1.75]{577878}{includetext height=0.5}{code39ext}}[12cm]
	\marginnote{\psbarcode[rotate=90,transx=1.75]{581027}{includetext height=0.5}{code39ext}}[17.5cm]
}

\newcommand*{\QED}{\hfill\ensuremath{q.e.d.}}
\newcommand{\ex}[1]{\newpage\subsubsection*{Aufgabe #1.}}
\newcommand*{\simpleTable}[2]{
	\begin{tabular}{@{} #1 @{}}
		\toprule
		#2
		\bottomrule
	\end{tabular}
}
\DeclareMathOperator{\sig}{sig}

\begin{document}
	%%%%%%%%%%%%%%%%%%%% Aufgabe 31 %%%%%%%%%%%%%%%%%%%%%%%%%%%%%%%%%%%%%%%%%%%%%%%%%
	\ex{31}
	
	Ausgangssignal:
	\[ s^T = (1\ 2\ 3\ 1\ -1\ -1\ -1\ 0) \]\\
	
	Diskrete Cosinustransformationsmatrix:
	
	
	\[\left(\begin{array}{rrrrrrrr}
	0.3536 & 0.3536 & 0.3536 & 0.3536 & 0.3536 & 0.3536 & 0.3536 & 0.3536 \\
	0.4904 & 0.4157 & 0.2778 & 0.0975 &-0.0975 &-0.2778 &-0.4157 &-0.4904 \\
	0.4619 & 0.1913 & -0.1913 & -0.4619 & -0.4619 & -0.1913 & 0.1913 & 0.4619 \\
	0.4157 & -0.0975 & -0.4904 & -0.2778 & 0.2778 & 0.4904 & 0.0975 & -0.4157 \\
	0.3536 & -0.3536 & -0.3536 & 0.3536 & 0.3536 & -0.3536 & -0.3536 & 0.3536 \\
	0.2778 & -0.4904 & 0.0975 & 0.4157 & -0.4157 & -0.0975 & 0.4904 & -0.2778 \\
	0.1913 & -0.4619 & 0.4619 & -0.1913 & -0.1913 & 0.4619 & -0.4619 & 0.1913 \\
	0.0975 & -0.2778 & 0.4157 & -0.4904 & 0.4904 & -0.4157 & 0.2778 & -0.0975
	\end{array}\right)\]\\
	
	Spektrum $s'$:
	\[s' = DCT \cdot s = (1.4142\ 3.0438\ 0.2706\ -2.3940\ -0.7071\ 0.0283\ 0.6533\ -0.0537)^T\]\\
	
	
	Rücktransformation mit $A = DWT^T$:
	\[ s = DCT^T \cdot s' = (1\ 2\ 3\ 1\ -1\ -1\ -1\ 0)^T \]\\
	
	$DCT^T=$
	\[\left(\begin{array}{rrrrrrrr}
	0.3536 & 0.4904 & 0.4619 & 0.4157 & 0.3536 & 0.2778 & 0.1913 & 0.0975 \\
	0.3536 & 0.4157 & 0.1913 & -0.0975 &-0.3536 &-0.4904 &-0.4619 &-0.2778 \\
	0.3536 & 0.2778 & -0.1913 & -0.4904 & -0.3536 & 0.0975 & 0.4619 & 0.4157 \\
	0.3536 & 0.0975 & -0.4619 & -0.2778 & 0.3536 & 0.4157 & -0.1913 & -0.4904 \\
	0.3536 & -0.0975 & -0.4619 & 0.2778 & 0.3536 & -0.4157 & -0.1913 & 0.4904 \\
	0.3536 & -0.2778 & -0.1913 & 0.4904 & -0.3536 & -0.0975 & 0.4619 & -0.4157 \\
	0.3536 & -0.4157 & 0.1913 & 0.0975 & -0.3536 & 0.4904 & -0.4619 & 0.2778 \\
	0.3536 & -0.4904 & 0.4619 & -0.4157 & 0.3536 & -0.2778 & 0.1913 & -0.0975
	\end{array}\right)\]
	
	%%%%%%%%%%%%%%%%%%%% Aufgabe 32 %%%%%%%%%%%%%%%%%%%%%%%%%%%%%%%%%%%%%%%%%%%%%%%%%
	\ex{32}
	
	Ausgangssignal:
	\[ s^T = (1\ 2\ 3\ 1\ -1\ -1\ -1\ 0) \]\\
	
	Diskrete Sinustransformationsmatrix:
	
	\[\left(\begin{array}{rrrrrrrr}
	0.1612 & 0.3030 & 0.4082 & 0.4642 & 0.4642 & 0.4082 & 0.3030 & 0.1612 \\
	0.3030 & 0.4642 & 0.4082 & 0.1612 &-0.1612 &-0.4082 &-0.4642 &-0.3030 \\
	0.4082 & 0.4082 & 0 & -0.4082 & -0.4082 & 0 & 0.4082 & 0.4082 \\
	0.4642 & 0.1612 & -0.4082 & -0.3030 & 0.3030 & 0.4082 & -0.1612 & -0.4642 \\
	0.4642 & -0.1612 & -0.4082 & 0.3030 & 0.3030 & -0.4082 & -0.1612 & 0.4642 \\
	0.4082 & -0.4082 & 0 & 0.4082 & -0.4082 & 0 & 0.4082 & -0.4082 \\
	0.3030 & -0.4642 & 0.4082 & -0.1612 & -0.1612 & 0.4082 & -0.4642 & 0.3030 \\
	0.1612 & -0.3030 & 0.4082 & -0.4642 & 0.4642 & -0.4082 & 0.3030 & -0.1612
	\end{array}\right)\]\\
	
	Spektrum $s'$:
	\[s' = DCT \cdot s = (1.2807\ 3.6512\ 0.8165\ -1.2911\ -0.5135\ 0\ 0.6553\ -0.0433)^T\]\\
	
	
	
	Rücktransformation mit $A = DST^T$:
	\[ s = DST^T \cdot s' = (1\ 2\ 3\ 1\ -1\ -1\ -1\ 0)^T \]\\
	
	$DST^T=$
	\[\left(\begin{array}{rrrrrrrr}
	0.1612 & 0.3030 & 0.4082 & 0.4642 & 0.4642 & 0.4082 & 0.3030 & 0.1612 \\
	0.3030 & 0.4642 & 0.4082 & 0.1612 &-0.1612 &-0.4082 &-0.4642 &-0.3030 \\
	0.4082 & 0.4082 & 0 & -0.4082 & -0.4082 & 0 & 0.4082 & 0.4082 \\
	0.4642 & 0.1612 & -0.4082 & -0.3030 & 0.3030 & 0.4082 & -0.1612 & -0.4642 \\
	0.4642 & -0.1612 & -0.4082 & 0.3030 & 0.3030 & -0.4082 & -0.1612 & 0.4642 \\
	0.4082 & -0.4082 & 0 & 0.4082 & -0.4082 & 0 & 0.4082 & -0.4082 \\
	0.3030 & -0.4642 & 0.4082 & -0.1612 & -0.1612 & 0.4082 & -0.4642 & 0.3030 \\
	0.1612 & -0.3030 & 0.4082 & -0.4642 & 0.4642 & -0.4082 & 0.3030 & -0.1612
	\end{array}\right)\]
	
	%%%%%%%%%%%%%%%%%%%% Aufgabe 33 %%%%%%%%%%%%%%%%%%%%%%%%%%%%%%%%%%%%%%%%%%%%%%%%%
	\ex{33}
	
	Ausgangssignal:
	\[ s^T = (1\ 2\ 3\ 1) \]\\
	
	
	
	
	
	
	\end{document}